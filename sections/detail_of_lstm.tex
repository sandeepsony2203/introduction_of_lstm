%!TEX root = ../Main.tex

\newcommand{\inlinesymbol}[2]{%
  \tikz[baseline = -0.7ex]{\node[#1, scale = 0.7] {#2};}%
}

\clearpage
\section{\lstm{} 详细介绍}
\subsection{\lstm{} 总体结构}
所有循环神经网络结构都是由完全相同结构的模块进行复制而成的.在普通的RNNs 中,这个模块结构非常简单,比如仅是一个单一的 \inlinesymbol{rectangle operator}{$\tanh$} 层,如\cref{fig:Inner Structure of RNN} 所示.%
%
\begin{figure}[!htb]
  \centering
  \scalebox{0.7}{\begin{tikzpicture}
  \rnnnode[position = {(0, 0)}, prefix = rnn1]{};
  \rnnnode[position = {(5.5, 0)}, prefix = rnn2]{};
  \rnnnode[position = {(11.0, 0)}, prefix = rnn3]{};

  \node[fit = (rnn1-frame.north west) (rnn1-frame.south east), cover] {};
  \node[fit = (rnn3-frame.north west) (rnn3-frame.south east), cover] {};
  \drawrnnborder[prefix = rnn1]{};
  \drawrnnborder[prefix = rnn3]{};

  \node[scale = 5] at (rnn1-frame.center) {\shadowtext{$A_{t-1}$}};
  \node[scale = 5] at (rnn3-frame.center) {\shadowtext{$A_{t+1}$}};

  \draw[flow, -] (rnn1-ht-1) -- ++(180:1cm) node[midway, above] {$\vec{h}_{t-2}$};
  \draw[flow, -] (rnn1-ht) -- (rnn2-ht-1)  node[midway, above] {$\vec{h}_{t-1}$};
  \draw[flow, -] (rnn2-ht) -- (rnn3-ht-1)  node[midway, above] {$\vec{h}_{t}$};
  \draw[flow]    (rnn3-ht) -- ++(0:1cm)     node[midway, above] {$\vec{h}_{t+1}$};

  \foreach \h/\x [count = \i] in {t-1/t-1, t/t, t+1/t+1}{%
    \draw[flow] (rnn\i-ht') -- ++(90:0.5cm) node[output, anchor = south] {$\vec{h}_{\h}$};
    \draw[flow, -] (rnn\i-xt) -- ++(-90:0.5cm) node[input, anchor = north] {$\vec{x}_{\x}$};
  }
\end{tikzpicture}}
  \caption{RNN 内部结构}
  \label{fig:Inner Structure of RNN}
\end{figure}

\lstm{} 也有类似的结构.但是每一个节点不再只是用一个单一的 \inlinesymbol{rectangle operator}{$\tanh$} 层,而是用了四个相互作用的层.\lstm{} 的内部结构\cref{fig:Inner Structure of LSTM} 所示,在 $t$ 时刻处理第 $t$ 个输入 $\vec{x}_t$ 时,需要上一时刻,也就是 $t-1$ 时刻的输出,即 $\vec{m}_{t-1}$ 和 $\vec{h}_{t-1}$.%
%
\begin{figure}[!htb]
  \centering
  \scalebox{0.7}{\begin{tikzpicture}
  \lstmnode[position = {(0, 5)}, prefix = lstm1]{};
  \lstmnode[position = {(7, 5)}, prefix = lstm2]{};
  \lstmnode[position = {(14, 5)}, prefix = lstm3]{};

  \node[fit = (lstm1-frame.north west) (lstm1-frame.south east), cover] {};
  \node[fit = (lstm3-frame.north west) (lstm3-frame.south east), cover] {};
  \drawlstmborder[prefix = lstm1]{};
  \drawlstmborder[prefix = lstm3]{};

  \node[scale = 6] at (lstm1-frame.center) {\shadowtext{$A_{t-1}$}};
  \node[scale = 6] at (lstm3-frame.center) {\shadowtext{$A_{t+1}$}};

  \draw[flow, -] (lstm1-mt-1) -- ++(180:1cm) node[midway, above] {$\vec{m}_{t-2}$};
  \draw[flow, -] (lstm1-ht-1) -- ++(180:1cm) node[midway, above] {$\vec{h}_{t-2}$};
  \draw[flow, -] (lstm1-mt) -- (lstm2-mt-1)  node[midway, above] {$\vec{m}_{t-1}$};
  \draw[flow, -] (lstm1-ht) -- (lstm2-ht-1)  node[midway, above] {$\vec{h}_{t-1}$};
  \draw[flow, -] (lstm2-mt) -- (lstm3-mt-1)  node[midway, above] {$\vec{m}_{t}$};
  \draw[flow, -] (lstm2-ht) -- (lstm3-ht-1)  node[midway, above] {$\vec{h}_{t}$};
  \draw[flow]    (lstm3-mt) -- ++(0:1cm)     node[midway, above] {$\vec{m}_{t+1}$};
  \draw[flow]    (lstm3-ht) -- ++(0:1cm)     node[midway, above] {$\vec{h}_{t+1}$};

  \foreach \h/\x [count = \i] in {t-1/t-1, t/t, t+1/t+1}{%
    \draw[flow] (lstm\i-ht') -- ++(90:0.5cm) node[output, anchor = south] {$\vec{h}_{\h}$};
    \draw[flow, -] (lstm\i-xt) -- ++(-90:0.5cm) node[input, anchor = north] {$\vec{x}_{\x}$};
  }
\end{tikzpicture}}
  \caption{\lstm{} 内部结构}
  \label{fig:Inner Structure of LSTM}
\end{figure}%

\cref{fig:Inner Structure of LSTM} 中的符合的含义如\cref{fig:Legend of Inner Structure of LSTM} 所示.%
%
\begin{figure}[!htb]
  \centering
  \scalebox{0.7}{\begin{tikzpicture}
  \node[rectangle operator] (sigmoid) {$\sigma$};
  \node[rectangle operator, anchor = west] (tanh) at ([xshift = 2pt]sigmoid.east) {$\tanh$};

  \node[circle operator, anchor = west] (times) at ([xshift = 1cm]tanh.east) {$\times$};
  \node[circle operator, anchor = west] (plus) at ([xshift = 2pt]times.east) {$+$};
  \node[ellipse operator, anchor = west] (tanh') at ([xshift = 2pt]plus.east) {$\tanh$};

  \draw[flow] ([xshift = 1cm]tanh'.east) coordinate(trans1) -- ++(0:1.5cm) coordinate (trans2);
  \coordinate(trans) at ($(trans1)!0.5!(trans2)$);

  \draw[flow] ([xshift = 1cm, yshift = 4pt]trans2) to[out = -15, in = 180] ([xshift = 2.5cm]trans2) coordinate (cat2);
  \draw[flow] ([xshift = 1cm, yshift = -4pt]trans2) to[out = 15, in = 180] ([xshift = 2.5cm]trans2);
  \coordinate (cat) at ($([xshift = 1cm]trans2)!0.5!(cat2)$);

  \draw[flow] ([xshift = 1cm]cat2) to[in = 195, out = 0] ([xshift = 2.5cm, yshift = 4pt]cat2);
  \draw[flow] ([xshift = 1cm]cat2) to[in = 165, out = 0] ([xshift = 2.5cm, yshift = -4pt]cat2);
  \coordinate (copy) at ([xshift = 1.75cm]cat2);

  \node[below = 0.5cm] at ($(sigmoid)!0.5!(tanh)$) {神经网络};
  \node[below = 0.5cm] at ($(times.west)!0.5!(tanh'.east)$) {逐点运算};
  \node[below = 0.5cm] at (trans) {传输};
  \node[below = 0.5cm] at (cat) {连接};
  \node[below = 0.5cm] at (copy) {复制};
\end{tikzpicture}}
  \caption{\cref{fig:Inner Structure of LSTM} 的符号说明}
  \label{fig:Legend of Inner Structure of LSTM}
\end{figure}%
%
可以看出对于一个 \lstm{} 节点来说,里面含有 $5$ 种类型元素.黑色的节点 \inlinesymbol{circle operator}{$\times$}、\inlinesymbol{circle operator}{$+$} 和 \inlinesymbol{ellipse operator}{$\tanh$} 表示向量逐位运算,假设向量 $\vec{\alpha} = (\alpha_1, \alpha_2, \cdots, \alpha_n)^\top$ 和向量 $\vec{\beta} = (\beta_1, \beta_2, \cdots, \beta_n)^\top$,且 $\vec{\alpha}\in\mathbb{R}^{n\times 1}$,$\vec{\beta}\in\mathbb{R}^{n\times 1}$,则向量 $\vec{\alpha}$ 和 $\vec{\beta}$ 的逐位运算定义如\cref{eqn:Pointwise Multiplication of Two Vectors}、\cref{eqn:Pointwise Addition of Two Vectors} 和\cref{eqn:Pointwise tanh of Vector} 所示.%
%
\begin{align}
  \label{eqn:Pointwise Multiplication of Two Vectors}
  \vec{\alpha}\cdot\vec{\beta} &= (\alpha_1\beta_1, \alpha_2\beta_2, \cdots, \alpha_n\beta_n)^\top\text{,}\\
  \label{eqn:Pointwise Addition of Two Vectors}
  \vec{\alpha}+\vec{\beta} &= (\alpha_1 + \beta_1, \alpha_2 + \beta_2, \cdots, \alpha_n + \beta_n)^\top\text{,}\\
  \label{eqn:Pointwise tanh of Vector}
  \tanh{\vec{\alpha}} &= (\tanh{\alpha_1}, \tanh{\alpha_2}, \cdots, \tanh{\alpha_n})^\top\text{.}
\end{align}%
%
蓝色节点 \inlinesymbol{rectangle operator}{$\sigma$} 和 \inlinesymbol{rectangle operator}{$\tanh$} 表示神经网络,其中 $\sigma$ 和 $\tanh$ 表示神经网络中的激活函数,这个会在以后详细说明.传输符号的输入输出的一样的,向量 $\vec{\alpha}$ 经过传输符号不会发生变化.连接符号会将两个向量连接起来,向量 $\vec{\alpha}$ 和 $\vec{\beta}$ 经过连接符号会变成一个向量 $[\vec{\alpha};\vec{\beta}] = (\alpha_1, \alpha_2, \cdots, \alpha_n, \beta_1, \beta_2, \cdots, \beta_n)^\top$,且不需要向量 $\vec{\alpha}$ 和向量 $\vec{\beta}$ 维数一样.而复制节点会将输入的向量变成两个相同的向量,即向量 $\vec{\alpha}$ 经过复制节点会变成两个相同的向量 $\vec{\alpha}$.

\subsection{\lstm{} 记忆结构}
我们可以看到,对于每个 \lstm{} 节点,其上部都有一个贯穿节点的数据流流向,如\cref{fig:State Transfer of LSTM} 所示,这个结构被称为记忆结构.%
%
\begin{figure}[!htb]
  \centering
  \scalebox{0.7}{\begin{tikzpicture}
  \lstmnode{};

  \coordinate (ht-1) at ([xshift = -1cm]lstm-ht-1);
  \coordinate (mt-1) at ([xshift = -1cm]lstm-mt-1);
  \coordinate (ht)   at ([xshift = 1cm]lstm-ht);
  \coordinate (mt)   at ([xshift = 1cm]lstm-mt);
  \draw[flow]    (lstm-ht) -- (ht) node[midway, above] {$\vec{h}_{t}$};
  \draw[flow, -] (lstm-ht-1) -- (ht-1) node[midway, above] {$\vec{h}_{t-1}$};
  \draw[flow]    (lstm-ht') -- ++(90:0.5cm) node[output, anchor = south] (ht') {$\vec{h}_{t}$};
  \draw[flow, -] (lstm-xt) -- ++(-90:0.5cm) node[input, anchor = north] (xt) {$\vec{x}_{t}$};

  \node[cover, fit = (xt.south -| ht-1) (ht |- ht'.north)] {};
  \drawlstmborder{};
  \draw[flow, -] (mt-1) -- (lstm-mt-1) node[midway, above] {$\vec{m}_{t-1}$};
  \draw[flow] (lstm-mt-1) -- (lstm-times1);
  \draw[flow] (lstm-times1) -- (lstm-plus);
  \draw[flow, -] (lstm-plus) -- (lstm-mt);
  \draw[flow] (lstm-mt) -- (mt) node[midway, above] {$\vec{m}_t$};

  \node[circle operator] at (lstm-times1) {$\times$};
  \node[circle operator] at (lstm-plus) {$+$};
\end{tikzpicture}}
  \caption{\lstm{} 节点的记忆结构}
  \label{fig:State Transfer of LSTM}
\end{figure}%
%
记忆的输入是上一个节点记忆向量 $\vec{m}_{t-1}\in\mathbb{R}^{m\times 1}$,记忆向量 $\vec{m}_{t-1}$ 经过一系列逐位运算会得到该 \lstm{} 节点记忆向量 $\vec{m}_t$.记忆结构使得实现 \lstm{} 节点的记忆保留,而且这种记忆保留是有选择性的,这就使得 \lstm{} 具有记忆的能力,因此,\lstm{} 可以被用来处理与序列相关的问题.

\subsection{\lstm{} 遗忘门结构}
但是,我们注意到了,记忆向量 $\vec{m}_t$ 是 \lstm{} 节点记忆保留的载体,而且,从头至尾只有逐位运算,其维度是不会发生变化的.这就意味着,固定长度的记忆向量 $\vec{m}_t$ 不可能无限制的对输入进行记忆,因此 \lstm{} 设计了遗忘门这个环节,实现有选择的忘掉某些旧的信息,只有这样,才能有新的信息加入.遗忘门的结构如\cref{fig:Forget Gate of LSTM} 所示.%
%
\begin{figure}[!htb]
  \centering
  \scalebox{0.7}{\begin{tikzpicture}
  \lstmnode{};

  \coordinate (ht-1) at ([xshift = -1cm]lstm-ht-1);
  \coordinate (mt-1) at ([xshift = -1cm]lstm-mt-1);
  \coordinate (ht)   at ([xshift = 1cm]lstm-ht);
  \coordinate (mt)   at ([xshift = 1cm]lstm-mt);
  \draw[flow]    (lstm-ht) -- (ht) node[midway, above] {$\vec{h}_{t}$};
  \draw[flow, -] (lstm-ht-1) -- (ht-1) node[midway, above] {$\vec{h}_{t-1}$};
  \draw[flow]    (lstm-mt) -- (mt) node[midway, above] {$\vec{m}_{t}$};
  \draw[flow, -] (lstm-mt-1) -- (mt-1) node[midway, above] {$\vec{m}_{t-1}$};
  \draw[flow]    (lstm-ht') -- ++(90:0.5cm) node[output, anchor = south] (ht') {$\vec{h}_{t}$};
  \draw[flow, -] (lstm-xt) -- ++(-90:0.5cm) node[input, anchor = north] (xt) {$\vec{x}_{t}$};

  \node[cover, fit = (xt.south -| ht-1) (ht |- ht'.north)] {};
  \drawlstmborder{};

  \node[input] at (xt) {$\vec{x}_{t}$};
  \draw[flow, -] (ht-1) -- (lstm-ht-1) node[midway, above] {$\vec{h}_{t-1}$};
  \draw[flow] (xt) |- (lstm-sigmoid1 |- lstm-ht-1) -- (lstm-sigmoid1);
  \draw[flow] (lstm-ht-1) -| (lstm-sigmoid1);
  \node[rectangle operator] at (lstm-sigmoid1) {$\sigma$};
  \draw[flow] (lstm-sigmoid1) -- (lstm-times1) node[midway, left] {$\vec{f}_t$};
\end{tikzpicture}}
  \caption{\lstm{} 节点的遗忘门结构}
  \label{fig:Forget Gate of LSTM}
\end{figure}%
%
可以看出,遗忘门的输入是向量 $\vec{h}_{t-1}\in\mathbb{R}^{m\times 1}$ 以及输入序列中第 $t$ 个元素所对应的特征向量 $\vec{x}_t\in\mathbb{R}^{n\times 1}$ 连接后的向量 $[\vec{h}_{t-1};\vec{x}_t]\in\mathbb{R}^{(m+n)\times 1}$,其输出是向量 $\vec{f}_t\in\mathbb{R}^{m\times 1}$.遗忘门的输入输出关系如\cref{eqn:Input Output of Forget Gate} 所示.%
%
\begin{equation}\label{eqn:Input Output of Forget Gate}
  \vec{f}_t = \sigma\big(\bm{W}_f[\vec{h}_{t-1};\vec{x}_t] + \vec{b}_f\big)\text{,}
\end{equation}%
%
其中 $\bm{W}_f\in\mathbb{R}^{m\times (m+n)}$ 是遗忘门的权重矩阵,$\vec{b}_f\in\mathbb{R}^{m\times 1}$ 是遗忘门的偏置向量,$\sigma$ 是 Sigmoid 函数,是遗忘门的激活函数,其表达式如\cref{eqn:Definition of Sigmoid Function} 所示.%
%
\begin{equation}\label{eqn:Definition of Sigmoid Function}
  \sigma(x) = \frac{1}{1+\exp(-x)}\text{.}
\end{equation}

\subsection{\lstm{} 输入门结构}
决定了遗忘什么,就是为了给新的输入腾出记忆空间.通过仔细对比\cref{fig:Inner Structure of RNN} 我们可以看出,\cref{fig:Input Gate of LSTM} 中 \lstm{} 节点中的 $\vec{m}'_t$ 就是传统 RNN 的节点输出 $\vec{h}_t$.只不过,在 \lstm{} 中,这个输入值不能一起送到输出口,\lstm{} 提出的输入门结构可以选择让 $\vec{m}'_t$ 中哪些信息通过,实现有选择的输入.\lstm{} 的输入门结构如\cref{fig:Input Gate of LSTM} 所示.%
%
\begin{figure}[!htb]
  \centering
  \scalebox{0.7}{\begin{tikzpicture}
  \lstmnode{};

  \coordinate (ht-1) at ([xshift = -1cm]lstm-ht-1);
  \coordinate (mt-1) at ([xshift = -1cm]lstm-mt-1);
  \coordinate (ht)   at ([xshift = 1cm]lstm-ht);
  \coordinate (mt)   at ([xshift = 1cm]lstm-mt);
  \draw[flow]    (lstm-ht) -- (ht) node[midway, above] {$\vec{h}_{t}$};
  \draw[flow, -] (lstm-ht-1) -- (ht-1) node[midway, above] {$\vec{h}_{t-1}$};
  \draw[flow]    (lstm-mt) -- (mt) node[midway, above] {$\vec{m}_{t}$};
  \draw[flow, -] (lstm-mt-1) -- (mt-1) node[midway, above] {$\vec{m}_{t-1}$};
  \draw[flow]    (lstm-ht') -- ++(90:0.5cm) node[output, anchor = south] (ht') {$\vec{h}_{t}$};
  \draw[flow, -] (lstm-xt) -- ++(-90:0.5cm) node[input, anchor = north] (xt) {$\vec{x}_{t}$};

  \node[cover, fit = (xt.south -| ht-1) (ht |- ht'.north)] {};
  \drawlstmborder{};

  \node[input] at (xt) {$\vec{x}_{t}$};
  \draw[flow, -] (ht-1) -- (lstm-ht-1) node[midway, above] {$\vec{h}_{t-1}$};
  \draw[flow] (xt) |- (lstm-sigmoid1 |- lstm-ht-1) -| (lstm-tanh1);
  \draw[flow] (lstm-ht-1) -| (lstm-sigmoid2);
  \node[rectangle operator] at (lstm-sigmoid2) {$\sigma$};
  \node[rectangle operator] at (lstm-tanh1) {$\tanh$};
  \draw[flow] (lstm-sigmoid2) |- (lstm-times2) node[midway, left] {$\vec{i}_t$};
  \draw[flow] (lstm-tanh1) -- (lstm-times2) node[midway, left] {$\vec{m}'_t$};
\end{tikzpicture}}
  \caption{\lstm{} 节点的输入门结构}
  \label{fig:Input Gate of LSTM}
\end{figure}%
%
可以看出,输入门的输入是向量 $\vec{h}_{t-1}$ 以及输入序列中第 $t$ 个元素所对应的特征向量 $\vec{x}_t$ 连接后的向量 $[\vec{h}_{t-1};\vec{x}_t]$,其输出是向量 $\vec{i}_t\in\mathbb{R}^{m\times 1}$、$\vec{m}'_t\in\mathbb{R}^{m\times 1}$.输入门的输入输出关系如\cref{eqn:Input Output of Input Gate i} 以及\cref{eqn:Input Output of Input Gate C} 所示.%
%
\begin{align}
  \label{eqn:Input Output of Input Gate i}
  \vec{i}_t  &= \sigma\big(\bm{W}_i[\vec{h}_{t-1};\vec{x}_t] + \vec{b}_i\big)\text{,}\\
  \label{eqn:Input Output of Input Gate C}
  \vec{m}'_t &= \tanh\big(\bm{W}_m[\vec{h}_{t-1};\vec{x}_t] + \vec{b}_m\big)\text{.}
\end{align}%
%
其中 $\bm{W}_i\in\mathbb{R}^{m\times (m+n)}$ 是输入门的权重矩阵,$\vec{b}_i\in\mathbb{R}^{m\times 1}$ 是输入门的偏置向量,$\bm{W}_m\in\mathbb{R}^{m\times (m+n)}$ 是输入权重矩阵,$\vec{b}_m\in\mathbb{R}^{m\times 1}$ 是输入偏置向量,$\tanh$ 是双曲正切函数,其表达式如\cref{eqn:Definition of tanh Function} 所示.%
%
\begin{equation}\label{eqn:Definition of tanh Function}
  \tanh(x) = \frac{\exp(x) - \exp(-x)}{\exp(x) + \exp(-x)}\text{.}
\end{equation}

\subsection{\lstm{} 更新记忆结构}
介绍完遗忘门和输入门,在这里可以对遗忘门和输入门进行总结,遗忘门计算向量 $\vec{f}_t$ 来选择性的保留记忆,而输入门则是计算向量 $\vec{i}_t$ 来选择性的记忆新的信息.有了这个总结,\lstm{} 的更新记忆结构便很好理解了,\lstm{} 的更新记忆结构如\cref{fig:Cell Update of LSTM} 所示.%
%
\begin{figure}[!htb]
  \centering
  \scalebox{0.7}{\begin{tikzpicture}
  \lstmnode{};

  \coordinate (ht-1) at ([xshift = -1cm]lstm-ht-1);
  \coordinate (mt-1) at ([xshift = -1cm]lstm-mt-1);
  \coordinate (ht)   at ([xshift = 1cm]lstm-ht);
  \coordinate (mt)   at ([xshift = 1cm]lstm-mt);
  \draw[flow]    (lstm-ht) -- (ht) node[midway, above] {$\vec{h}_{t}$};
  \draw[flow, -] (lstm-ht-1) -- (ht-1) node[midway, above] {$\vec{h}_{t-1}$};
  \draw[flow]    (lstm-mt) -- (mt) node[midway, above] {$\vec{m}_{t}$};
  \draw[flow, -] (lstm-mt-1) -- (mt-1) node[midway, above] {$\vec{m}_{t-1}$};
  \draw[flow]    (lstm-ht') -- ++(90:0.5cm) node[output, anchor = south] (ht') {$\vec{h}_{t}$};
  \draw[flow, -] (lstm-xt) -- ++(-90:0.5cm) node[input, anchor = north] (xt) {$\vec{x}_{t}$};

  \node[cover, fit = (xt.south -| ht-1) (ht |- ht'.north)] {};
  \drawlstmborder{};

  \draw[flow, -] (mt-1) -- (lstm-mt-1) node[midway, above] {$\vec{m}_{t-1}$};
  \draw[flow] (lstm-sigmoid1) -- (lstm-times1) node[midway, left] {$\vec{f}_t$};
  \draw[flow] (lstm-sigmoid2) |- (lstm-times2) node[midway, left] {$\vec{i}_t$};
  \draw[flow] (lstm-tanh1) -- (lstm-times2) node[midway, left] {$\vec{m}'_t$};
  \draw[flow] (lstm-mt-1) -- (lstm-times1);
  \draw[flow] (lstm-times1) -- (lstm-plus);
  \draw[flow, -] (lstm-plus) -- (lstm-mt);
  \draw[flow] (lstm-mt) -- (mt) node[midway, above] {$\vec{m}_t$};
  \draw[flow] (lstm-times2) -- (lstm-plus);
  \node[circle operator] at (lstm-times1) {$\times$};
  \node[circle operator] at (lstm-times2) {$\times$};
  \node[circle operator] at (lstm-plus) {$+$};
\end{tikzpicture}}
  \caption{\lstm{} 节点更新记忆结构}
  \label{fig:Cell Update of LSTM}
\end{figure}%
%
由\cref{fig:Cell Update of LSTM} 可以看出,\lstm{} 节点在更新记忆的时候采取的策略很直接,就是将遗忘门的输出和输入门的输出结合起来,具体的运算如\cref{eqn:Memory Update of LSTM Node} 所示.%
%
\begin{equation}\label{eqn:Memory Update of LSTM Node}
  \vec{m}_t = \vec{f}_i\cdot\vec{m}_{t-1} + \vec{i}_t\cdot\vec{m}'_t\text{,}
\end{equation}

\subsection{\lstm{} 输出门结构}
其实向量 $\vec{m}_t$ 并不是 \lstm{} 节点的输出,向量 $\vec{m}_t$ 的作用只是辅助实现选择性遗忘和选择性记忆的效果.\lstm{} 节点的输出是 $\vec{h}_t$,输出 $\vec{h}_t$ 主要依赖于 $\vec{m}_t$,但是又不仅仅依赖于 $\vec{m}_t$,他需要在 $\vec{m}_t$ 的基础上进行过滤,这个过滤用的向量 $\vec{o}_t$ 是由向量 $\vec{h}_{t-1}$ 以及输入序列中第 $t$ 个元素所对应的特征向量 $\vec{x}_t$ 共同决定的.其更新的结构如\cref{fig:Cell Output of LSTM} 所示.%
%
\begin{figure}[!htb]
  \centering
  \scalebox{0.7}{\begin{tikzpicture}
  \lstmnode{};

  \coordinate (ht-1) at ([xshift = -1cm]lstm-ht-1);
  \coordinate (mt-1) at ([xshift = -1cm]lstm-mt-1);
  \coordinate (ht)   at ([xshift = 1cm]lstm-ht);
  \coordinate (mt)   at ([xshift = 1cm]lstm-mt);
  \draw[flow]    (lstm-ht) -- (ht) node[midway, above] {$\vec{h}_{t}$};
  \draw[flow, -] (lstm-ht-1) -- (ht-1) node[midway, above] {$\vec{h}_{t-1}$};
  \draw[flow]    (lstm-mt) -- (mt) node[midway, above] {$\vec{m}_{t}$};
  \draw[flow, -] (lstm-mt-1) -- (mt-1) node[midway, above] {$\vec{m}_{t-1}$};
  \draw[flow]    (lstm-ht') -- ++(90:0.5cm) node[output, anchor = south] (ht') {$\vec{h}_{t}$};
  \draw[flow, -] (lstm-xt) -- ++(-90:0.5cm) node[input, anchor = north] (xt) {$\vec{x}_{t}$};

  \node[cover, fit = (xt.south -| ht-1) (ht |- ht'.north)] {};
  \drawlstmborder{};

  \node[input] at (xt) {$\vec{x}_{t}$};
  \draw[flow, -] (ht-1) -- (lstm-ht-1) node[midway, above] {$\vec{h}_{t-1}$};
  \draw[flow] (xt) |- (lstm-sigmoid1 |- lstm-ht-1) -| (lstm-sigmoid3);
  \draw[flow] (lstm-ht-1) -| (lstm-sigmoid3);
  \node[rectangle operator] at (lstm-sigmoid3) {$\sigma$};
  \draw[flow] (lstm-sigmoid3) |- (lstm-times3) node[midway, left] {$\vec{o}_t$};
  \node[circle operator] at (lstm-times3) {$\times$};
  \node[ellipse operator] at (lstm-tanh2) {$\tanh$};
  \draw[flow] (lstm-plus) -| (lstm-tanh2);
  \draw[flow] (lstm-tanh2) -- (lstm-times3);
  \draw[flow, -] (lstm-times3) |- (lstm-ht);
  \draw[flow] (lstm-ht) -- (ht) node[midway, above] {$\vec{h}_t$};
  \draw[flow, -] (lstm-times3) |- (ht' |- ht) -- ([yshift = -3pt]ht' |- mt);
  \draw[flow] ([yshift = 3pt]ht' |- mt) -- (ht');
  \node[output] at (ht') {$\vec{h}_t$};
\end{tikzpicture}}
  \caption{\lstm{} 节点的输出门结构}
  \label{fig:Cell Output of LSTM}
\end{figure}%
%
由\cref{fig:Cell Output of LSTM} 我们可以得到输出门过滤向量 $\vec{o}_t$ 以及输出结果 $\vec{m}_t$ 的表达式,分别如\cref{eqn:Filter Vector of Output Gate} 和\cref{eqn:Output of Output Gate} 所示.%
%
\begin{align}
  \label{eqn:Filter Vector of Output Gate}
  \vec{o}_t &= \sigma\big(\bm{W}_o[\vec{h}_{t-1};\vec{x}_t] + \vec{b}_o\big)\text{,}\\
  \label{eqn:Output of Output Gate}
  \vec{h}_t &= \vec{o}_t\cdot\tanh(\vec{m}_t)\text{.}
\end{align}%
%
其中 $\vec{o}_t\in\mathbb{R}^{m\times 1}$ 是输出门过滤向量,$\bm{W}_o\in\mathbb{R}^{m\times (m+n)}$ 是输出门权重矩阵,$\vec{b}_o\in\mathbb{R}^{m\times 1}$ 是输出门偏置向量.

\subsection{小节}
\indent\lstm{} 是对传统 RNN 的改进,解决了传统 RNN 无法处理长期依赖预测的问题.\lstm{} 是通过在传统 RNN 中添加遗忘门以及输入门,配合辅助记忆向量 $\vec{m}_t$ 使得 \lstm{} 中的节点保留了对早期输入的记忆.本节介绍的 \lstm{} 节点计算公式中的权重矩阵 $\bm{W}_f$、$\bm{W}_i$、$\bm{W}_m$ 和 $\bm{W}_o$,以及偏置向量 $\vec{b}_f$、$\vec{b}_i$、$\vec{b}_m$ 和 $\vec{b}_o$ 都是 \lstm{} 的训练参数,需要利用大量的样本数据进行训练.